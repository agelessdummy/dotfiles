%%
%% This is file `schiller.tex',
%% generated with the docstrip utility.
%%
%% The original source files were:
%%
%% dramatist.dtx  (with options: `schiller')
%% 
%% IMPORTANT NOTICE:
%% 
%% For the copyright see the source file.
%% 
%% Any modified versions of this file must be renamed
%% with new filenames distinct from schiller.tex.
%% 
%% For distribution of the original source see the terms
%% for copying and modification in the file dramatist.dtx.
%% 
%% This generated file may be distributed as long as the
%% original source files, as listed above, are part of the
%% same distribution. (The sources need not necessarily be
%% in the same archive or directory.)
%% dramatist.dtx
%% Copyright (C) 2003-2005 Massimiliano Dominici
%% \CharacterTable%%  {Upper-case    \A\B\C\D\E\F\G\H\I\J\K\L\M\N\O\P\Q\R\S\T\U\V\W\X\Y\Z
%%   Lower-case    \a\b\c\d\e\f\g\h\i\j\k\l\m\n\o\p\q\r\s\t\u\v\w\x\y\z
%%   Digits        \0\1\2\3\4\5\6\7\8\9
%%   Exclamation   \!     Double quote  \"     Hash (number) \#
%%   Dollar        \$     Percent       \%     Ampersand     \&
%%   Acute accent  \'     Left paren    \(     Right paren   \)
%%   Asterisk      \*     Plus          \+     Comma         \,
%%   Minus         \-     Point         \.     Solidus       \/
%%   Colon         \:     Semicolon     \;     Less than     \<
%%   Equals        \=     Greater than  \>     Question mark \?
%%   Commercial at \@     Left bracket  \[     Backslash     \\
%%   Right bracket \]     Circumflex    \^     Underscore    \_
%%   Grave accent  \`     Left brace    \{     Vertical bar  \|
%%   Right brace   \}     Tilde         \~}
%%
%%\section{The GNU General Public License}
%% This file shows some basic features of package `dramatist', when
%% typesetting a drama in prose. See the commented line for more
%% informations.
%%
%% The source for this example is taken from Schiller's `The
%% Robbers'.

\documentclass{book}
\usepackage{dramatist}

\pagestyle{plain}
%% Maybe you want acts and scenes print their marks in the headings.
%% The following lines should work.
%%\makeatletter
%%\def\ps@myheadings{%%
%%    \renewcommand\drampermark[1]{\markboth{##1}{}}
%%    \renewcommand\actmark[1]{\markboth{##1}{}}
%%    \renewcommand\scenemark[1]{\markright{##1}}
%%    \def\@oddfoot{\hfil\thepage\hfil}
%%    \def\@evenfoot{\hfil\thepage\hfil}
%%    \def\@evenhead{\hfil\scshape\leftmark\hfil}%%
%%    \def\@oddhead{\hfil\scshape\rightmark\hfil}%%
%%}
%%\makeatother
%%\pagestyle{myheadings}

%% We may change some parameters in the look of acts and scenes:
%%\renewcommand{\actnamefont}{\bfseries\Large}
%%\renewcommand{\theact}{\Roman{act}}
%%\renewcommand{\scenenamefont}{\bfseries\large}
%%\renewcommand{\thescene}{\Roman{scene}}

%% We may change some parameters in the look of characters:
%%\renewcommand{\castfont}{\bfseries}
%%\renewcommand{\speaksfont}{\itshape}
%%\renewcommand{\speaksdel}{.\\}
%%\renewcommand{\namefont}{\bfseries}

%% We may change some parameters in the look of stage directions:
%%\StageDirConf{\begin{center}\begin{minipage}{.7\textwidth}\bfseries\centering}{\end{minipage}\end{center}}
%%\renewcommand{\dirdelimiter}[1]{[#1]}

%% If you want the name of character printed centered above the
%% dialogue, you should comment out the following lines.
%%\Dlabelsep=0pt
%%\renewcommand{\speakslabel}[1]{%%
%%    \hbox to\textwidth{\hfill\speaksfont{#1}\hfill}}

\author{Friederich Schiller}
\title{The Robbers}
\date{}

\begin{document}
\begin{titlepage}
\maketitle
\end{titlepage}

\tableofcontents

%% We define some characters appearing in the play.
\Character[MAXIMILIAN, COUNT VON MOOR.]{Old Moor}{moor}
%% group of characters.
\begin{CharacterGroup}{his Sons.}
\GCharacter{CHARLES,}{Charles Von Moor}{cha}
\GCharacter{FRANCIS,}{Francis}{fran}
\end{CharacterGroup}
\Character[AMELIA VON EDELREICH, his Niece.]{Amelia}{amelia}
%% group of characters.
\begin{CharacterGroup}{Libertines, afterwards Banditti}
\GCharacter{SPIEGELBERG,}{Spiegelberg}{spie}
\GCharacter{SCHWEITZER,}{Schweitzer}{schwei}
\GCharacter{GRIMM,}{Grimm}{grim}
\GCharacter{RAZMANN,}{Razmann}{raz}
\GCharacter{SCHUFTERLE,}{Schufterle}{schuf}
\GCharacter{ROLLER,}{Roller}{rol}
\GCharacter{KOSINSKY,}{Kosinsky}{kos}
\GCharacter{SCHWARTZ,}{Schwartz}{schwa}
\end{CharacterGroup}
\Character[HERMANN, the natural son of a Nobleman.]{Hermann}{her}
\Character[DANIEL, an old Servant of Count von Moor.]{Daniel}{dan}
\Character[PASTOR MOSER.]{Pastor Moser}{pm}
\Character[FATHER DOMINIC, a Monk.]{Father Dominic}{fd}
%% the following collective character appears in the play as single
%% instances, so we don't need define commands and entries for it.
\Character[BAND OF ROBBERS, SERVANTS, ETC.]{}{}

%% We call the dramatis personae list.
\DramPer

\act

%% Schiller puts some general informations about location in the
%% scene heading. So, we must use uppercase version of `\scene'. On
%% the other hand we don't want this information to appear in
%% headers and table of contents; hence the empty optional argument.
\Scene[]{. -- Franconia}

\StageDir{Apartment in the Castle of COUNT MOOR.\\\fran, \moor}

\begin{drama}
\franspeaks But are you really well, father? You look so pale.

\moorspeaks Quite well, my son -- what have you to tell me?

\franspeaks The post is arrived -- a letter from our correspondent at
Leipsic.

\moorspeaks \direct{eagerly} Any tidings of my son Charles?

\franspeaks Hem! Hem! -- Why, yes. But I fear -- I know not -- whether I dare
 -- your health. -- Are you really quite well, father?

\moorspeaks As a fish in water. Does he write of my son? What means this
anxiety about my health? You have asked me that question twice.

\franspeaks If you are unwell -- or are the least apprehensive of being so --
permit me to defer -- I will speak to you at a fitter season. -- \direct{Half
aside.} These are no tidings for a feeble frame.

\moorspeaks Gracious Heavens? what am I doomed to hear?

\franspeaks First let me retire and shed a tear of compassion for my lost
brother. Would that my lips might be forever sealed -- for he is your
son! Would that I could throw an eternal veil over his shame -- for he is
my brother! But to obey you is my first, though painful, duty -- forgive
me, therefore.

\moorspeaks Oh, Charles! Charles! Didst thou but know what thorns thou
plantest in thy father's bosom! That one gladdening report of thee would
add ten years to my life! yes, bring back my youth! whilst now, alas,
each fresh intelligence but hurries me a step nearer to the grave!

\franspeaks Is it so, old man, then farewell! for even this very day we
might all have to tear our hair over your coffin.

\moorspeaks Stay! There remains but one short step more -- let him have his
will! \direct{He sits down.} The sins of the father shall be visited unto the
third and fourth generation -- let him fulfil the decree.

\franspeaks \direct{takes the letter out of his pocket}. You know our
correspondent! See! I would give a finger of my right hand might I
pronounce him a liar -- a base and slanderous liar! Compose yourself!
Forgive me if I do not let you read the letter yourself. You cannot,
must not, yet know all.

\moorspeaks All, all, my son. You will but spare me crutches.

\franspeaks \direct{reads} ``Leipsic, May 1. Were I not bound by an inviolable
promise to conceal nothing from you, not even the smallest particular,
that I am able to collect, respecting your brother's career, never, my
dearest friend, should my guiltless pen become an instrument of torture
to you. I can gather from a hundred of your letters how tidings such as
these must pierce your fraternal heart. It seems to me as though I saw
thee, for the sake of this worthless, this detestable'' -- \direct{\moor covers
his face}. Oh! my father, I am only reading you the mildest passages --
``this detestable man, shedding a thousand tears.'' Alas! mine flowed -- ay,
gushed in torrents over these pitying cheeks. ``I already picture to
myself your aged pious father, pale as death.'' Good Heavens! and so you
are, before you have heard anything.

\moorspeaks Go on! Go on!
\end{drama}

\begin{center}
[\dots]
\end{center}

\Scene[]{. -- A Tavern on the Frontier of Saxony.}

\StageDir{\cha intent on a book; \spie drinking at the table.}

\begin{drama}
\chaspeaks \direct{lays the book aside}. I am disgusted with this age of
puny scribblers when I read of great men in my Plutarch.

\spiespeaks \direct{places a glass before him, and drinks.} Josephus is the book
you should read.

\chaspeaks The glowing spark of Prometheus is burnt out, and now
they substitute for it the flash of lycopodium, a stage-fire which will
not so much as light a pipe. The present generation may be compared to
rats crawling about the club of Hercules.

A French abbe lays it down that Alexander was a poltroon; a phthisicky
professor, holding at every word a bottle of sal volatile to his nose,
lectures on strength. Fellows who faint at the veriest trifle criticise
the tactics of Hannibal; whimpering boys store themselves with phrases
out of the slaughter at Canna; and blubber over the victories of Scipio,
because they are obliged to construe them.

\spiespeaks Spouted in true Alexandrian style.

\chaspeaks A brilliant reward for your sweat in the battle-field
truly to have your existence perpetuated in gymnasiums, and your
immortality laboriously dragged about in a schoolboy's satchel. A
precious recompense for your lavished blood to be wrapped round
gingerbread by some Nuremberg chandler, or, if you have great luck, to
be screwed upon stilts by a French playwright, and be made to move on
wires! Ha, ha, ha!

\spiespeaks \direct{drinks.} Read Josephus, I tell you.

\chaspeaks Fie! fie upon this weak, effeminate age, fit for nothing
but to ponder over the deeds of former times, and torture the heroes of
antiquity with commentaries, or mangle them in tragedies. The vigor of
its loins is dried up, and the propagation of the human species has
become dependent on potations of malt liquor.
\end{drama}

\begin{center}
[\dots]
\end{center}

\act

\Scene[]{. -- {\scshape Francis von Moor} in his chamber -- in meditation.}

\begin{drama}
\franspeaks It lasts too long -- and the doctor even says is recovering -- an
old man's life is a very eternity! The course would be free and plain
before me, but for this troublesome, tough lump of flesh, which, like
the infernal demon-hound in ghost stories, bars the way to my treasures.

Must, then, my projects bend to the iron yoke of a mechanical system?
Is my soaring spirit to be chained down to the snail's pace of matter?
To blow out a wick which is already flickering upon its last drop of
oil -- 'tis nothing more. And yet I would rather not do it myself, on
account of what the world would say. I should not wish him to be
killed, but merely disposed of. I should like to do what your clever
physician does, only the reverse way -- not stop Nature's course by
running a bar across her path, but only help her to speed a little
faster. Are we not able to prolong the conditions of life? Why,
then, should we not also be able to shorten them? Philosophers and
physiologists teach us how close is the sympathy between the emotions of
the mind and the movements of the bodily machine. Convulsive sensations
are always accompanied by a disturbance of the mechanical vibrations --
passions injure the vital powers -- an overburdened spirit bursts its
shell. Well, then -- what if one knew how to smooth this unbeaten path,
for the easier entrance of death into the citadel of life? -- to work the
body's destruction through the mind -- ha! an original device! -- who can
accomplish this? -- a device without a parallel! Think upon it, Moor!
That were an art worthy of thee for its inventor. Has not poisoning
been raised almost to the rank of a regular science, and Nature
compelled, by the force of experiments, to define her limits, so that
one may now calculate the heart's throbbings for years in advance, and
say to the beating pulse, ``So far, and no farther''? Why should not one
try one's skill in this line?

And how, then, must I, too, go to work to dissever that sweet and
peaceful union of soul and body? What species of sensations should I
seek to produce? Which would most fiercely assail the condition of
life? Anger? -- that ravenous wolf is too quickly satiated. Care? that
worm gnaws far too slowly. Grief? -- that viper creeps too lazily for me.
Fear? -- hope destroys its power. What! and are these the only
executioners of man? is the armory of death so soon exhausted? \direct{In deep
thought.} How now! what! ho! I have it! \direct{Starting up.} Terror! What
is proof against terror? What powers have religion and reason under
that giant's icy grasp! And yet -- if he should withstand even this
assault? If he should! Oh, then, come Anguish to my aid! and thou,
gnawing Repentance! -- furies of hell, burrowing snakes who regorge your
food, and feed upon your own excrements; ye that are forever destroying,
and forever reproducing your poison! And thou, howling Remorse, that
desolatest thine own habitation, and feedest upon thy mother. And come
ye, too, gentle Graces, to my aid; even you, sweet smiling Memory,
goddess of the past -- and thou, with thy overflowing horn of plenty,
blooming Futurity; show him in your mirror the joys of Paradise, while
with fleeting foot you elude his eager grasp. Thus will I work my
battery of death, stroke after stroke, upon his fragile body, until the
troop of furies close upon him with Despair! Triumph! triumph! -- the
plan is complete -- difficult and masterly beyond compare -- sure -- safe; for
then \direct{with a sneer} the dissecting knife can find no trace of wound or
of corrosive poison.

\direct{Resolutely.} Be it so! \direct{Enter \her.} Ha! \emph{Deus ex machina}!
Hermann!

\herspeaks At your service, gracious sir!

\franspeaks \direct{shakes him by the hand.} You will not find it that of an
ungrateful master.

\herspeaks I have proofs of this.

\franspeaks And you shall have more soon -- very soon, Hermann! -- I have
something to say to thee, Hermann.

\herspeaks I am all attention.
\end{drama}

\begin{center}
[\dots]
\end{center}

\end{document}

\endinput
%%
%% End of file `schiller.tex'.
